\documentclass{beamer}

\newcommand{\course}{CS 1331 Introduction to Object Oriented Programming}
\newcommand{\lesson}{Review 1: Java Programs, Variables, Values, Control Structures, Arrays, Classes}
\newcommand{\code}{http://www.cc.gatech.edu/~simpkins/teaching/gatech/cs1331/code}

\author[Chris Simpkins]
{Christopher Simpkins \\\texttt{chris.simpkins@gatech.edu}}
\institute[Georgia Tech] % (optional, but mostly needed)

\date[CS 1331]{}
\subject{\lesson}


\newcommand{\course}{Introduction to Object-Oriented Programming}
\subject{\course}
\title[\lesson]{\course}
\subtitle{\lesson}

\author[CS 1331]
{Christopher Simpkins \\\texttt{chris.simpkins@gatech.edu}}
\institute[Georgia Tech]

\date[]{}

\newcommand{\link}[2]{\href{#1}{\textcolor{blue}{\underline{#2}}}}
\newcommand{\code}{http://www.cs1331.org/code}

\usepackage{colortbl}

% If you have a file called "university-logo-filename.xxx", where xxx
% is a graphic format that can be processed by latex or pdflatex,
% resp., then you can add a logo as follows:

% \pgfdeclareimage[width=0.6in]{coc-logo}{cc_2012_logo}
% \logo{\pgfuseimage{coc-logo}}

\mode<presentation>
{
  \usetheme{Berlin}
  \useoutertheme{infolines}

  % or ...

 \setbeamercovered{transparent}
  % or whatever (possibly just delete it)
}

\usepackage{tikz}
% Optional PGF libraries
\usepackage{pgflibraryarrows}
\usepackage{pgflibrarysnakes}
\usepackage{pgfplots}
\usepackage{fancybox}
\usepackage{listings}
\usepackage{hyperref}
\hypersetup{colorlinks=true,urlcolor=blue}
\usepackage[english]{babel}
% or whatever

\usepackage[latin1]{inputenc}
% or whatever

\usepackage{times}
\usepackage[T1]{fontenc}
% Or whatever. Note that the encoding and the font should match. If T1
% does not look nice, try deleting the line with the fontenc.


\usepackage{listings}

% "define" Scala
\lstdefinelanguage{scala}{
  morekeywords={abstract,case,catch,class,def,%
    do,else,extends,false,final,finally,%
    for,if,implicit,import,match,mixin,%
    new,null,object,override,package,%
    private,protected,requires,return,sealed,%
    super,this,throw,trait,true,try,%
    type,val,var,while,with,yield},
  otherkeywords={=>,<-,<\%,<:,>:,\#,@},
  sensitive=true,
  morecomment=[l]{//},
  morecomment=[n]{/*}{*/},
  morestring=[b]",
  morestring=[b]',
  morestring=[b]""",
}

\usepackage{color}
\definecolor{dkgreen}{rgb}{0,0.6,0}
\definecolor{gray}{rgb}{0.5,0.5,0.5}
\definecolor{mauve}{rgb}{0.58,0,0.82}

% Default settings for code listings
\lstset{frame=tb,
  language=scala,
  aboveskip=2mm,
  belowskip=2mm,
  showstringspaces=false,
  columns=flexible,
  basicstyle={\scriptsize\ttfamily},
  numbers=none,
  numberstyle=\tiny\color{gray},
  keywordstyle=\color{blue},
  commentstyle=\color{dkgreen},
  stringstyle=\color{mauve},
  frame=single,
  breaklines=true,
  breakatwhitespace=true,
  keepspaces=true
  %tabsize=3
}


% \beamerdefaultoverlayspecification{<+->}


\begin{document}

\begin{frame}
  \titlepage
\end{frame}


%------------------------------------------------------------------------
\begin{frame}[fragile]{The Anatomy of a Java Program}


\vspace{-.1in}
\begin{lstlisting}[language=Java,textsize=8pt]
import java.util.Random;

public class Hello {

    private static final String[] GREETINGS =
        {"Hello, world!", "Hi there!", "W'sup!"};

    private String greeting;

    private Hello() {
        Random rand = new Random();
        int greetingsIndex = rand.nextInt(GREETINGS.length);
        greeting = GREETINGS[greetingsIndex];
    }

    public Hello(int anIndex) {
        int greetingsIndex = anIndex % GREETINGS.length;
        greeting = GREETINGS[greetingsIndex];
    }

    public String getGreeting() {
        return greeting;
    }

    public static void main(String[] args) {
        Hello h = new Hello();
        System.out.println(h.getGreeting());
    }
}
\end{lstlisting}

\end{frame}
%------------------------------------------------------------------------

%------------------------------------------------------------------------
\begin{frame}[fragile]{The {\tt import} Statement}


The statement
\begin{lstlisting}[language=Java]
import java.util.Random;
\end{lstlisting}
allows us to use the {\tt java.util.Random} class in our program.  Without the import, the line
\begin{lstlisting}[language=Java]
  Random rand = new Random(1);
\end{lstlisting}
would prodiuce the following error:
\begin{lstlisting}[language=Java]
Hello.java:11: error: cannot find symbol
        Random rand = new Random(1);
        ^
  symbol:   class Random
  location: class Hello
\end{lstlisting}

\end{frame}
%------------------------------------------------------------------------

%------------------------------------------------------------------------
\begin{frame}[fragile]{The {\tt class} Declaration}


A class is declared with the following syntax:
\begin{lstlisting}[language=Java]
public class Hello { ... }
\end{lstlisting}

\begin{itemize}
\item {\tt public} means that the class is visible to any object within the class's package, or within any Java file that imports the class
\item {\tt class} means we're declaring and defining a class
\item the code between \{ and \} define the class
\item Every Java source file must contain exactly one public class.  In this case {\tt Hello} is public, so it must be in a file named {\tt Hello.java}
\end{itemize}


\end{frame}
%------------------------------------------------------------------------

%------------------------------------------------------------------------
\begin{frame}[fragile]{{\tt static} and {\tt final} Variables}

\vspace{-.1in}
\begin{lstlisting}[language=Java]
private static final String[] greetings =
    {"Hello, world!", "Hi there!", "W'sup!"};
\end{lstlisting}
\vspace{-.1in}
\begin{itemize}
\item {\tt private} means the variable is only visible within instances of the {\tt Hello} class
\item {\tt static} means that there is exactly one copy of this variable for all instances of the {\tt Hello} class, even if no {\tt Hello} objects have been instantiated
\item {\tt final} means we can only assign a value to this variable once.  After that, its value is constant.  Note that any variable can be {\tt final}, not just {\tt static} variables.
\item {\tt String[]} is the variable's type (in particular, a String array).  Every variable has a type, which restricts the values it may be assigned.
\item {\tt greetings} is the variable's name
\item In this case, we have both declared the variable {\tt greetings} and initialized it with the value {\tt \{"Hello, world!", "Hi there!", "W'sup!"\}}
\end{itemize}


\end{frame}
%------------------------------------------------------------------------

%------------------------------------------------------------------------
\begin{frame}[fragile]{Instance Variables}


\begin{lstlisting}[language=Java]
  private String greeting;
\end{lstlisting}
is an instance variable declaration.
\begin{itemize}
\item {\tt private} means it is only visible within the {\tt Hello} class
\item {\tt String} is the variables type
\item {\tt greeting} is the variable's name
\item Every {\tt Hello} object has it's own {\tt greeting} instance variable
\item Since we didn't initialize {\tt greeting}, its value is {\tt null} (until we execute a constructor)
\end{itemize}


\end{frame}
%------------------------------------------------------------------------

%------------------------------------------------------------------------
\begin{frame}[fragile]{No-arg Constructors}


\begin{lstlisting}[language=Java]
public Hello() {
    Random rand = new Random(10);
    int greetingsIndex = rand.nextInt(greetings.length);
    greeting = greetings[greetingsIndex];
}
\end{lstlisting}

\begin{itemize}
\item Constructors initialize an object of a class.  In this case, the constructor initializes the {\tt greeting} instance variable.
\item Constructors are called with operator {\tt new}, as in {\tt Hello h = new Hello();}
\item After {\tt Hello h = new Hello();}, {\tt h} holds the address of a {\tt Hello} object which has some randomly assigned {\tt greeting} instance variable.
\item {\tt h} is a {\it reference} to a {\tt Hello} object
\end{itemize}

What does {\tt Random rand = new Random(10);} do?

\end{frame}
%------------------------------------------------------------------------

%------------------------------------------------------------------------
\begin{frame}[fragile]{Constructors With Parameters}


\begin{lstlisting}[language=Java]
public Hello(int anIndex) {
    int greetingsIndex = anIndex % greetings.length;
    greeting = greetings[greetingsIndex];
}
\end{lstlisting}

\begin{itemize}
\item To call this constructor, provide an argument in the call to {\tt new}, as in {\tt Hello h = new Hello(1);}
\item Providing the argument in the {\tt new} call selects a particular constructor, in this case the constructor that takes one {\tt int} parameter
\end{itemize}


\end{frame}
%------------------------------------------------------------------------

%------------------------------------------------------------------------
\begin{frame}[fragile]{Getter Methods}

\vspace{-.1in}
\begin{lstlisting}[language=Java]
public String getGreeting() {
    return greeting;
}
\end{lstlisting}
\vspace{-.1in}
is a ``getter'' method, a.k.a., an ``accessor''.
\begin{itemize}
\item {\tt getGreeting()} returns the value of the {\tt greeting} instance variable for the object on which it is invoked
\item In general, the preferred way of naming getters is {\tt getMyVariable}, where {\tt myVariable} is an instance variable.
\item For boolean instance variabes, the naming convention is {\tt isMyFlag}, where {\tt myFlag} is a {\tt boolean} instance variable.
\item Using this naming convention makes a class usable as a Java Bean (just get in the naming habit for now)
\item ``Setters'', a.k.a. ``mutators'', are named the same way, e.g., {\tt setGreeting(String aNewGreeting)}
\item Since we don't have any setters in our {\tt Hello} class, instances of {\tt Hello} are immutable
\end{itemize}


\end{frame}
%------------------------------------------------------------------------

%------------------------------------------------------------------------
\begin{frame}[fragile]{The {\tt main} Method}


The method
\begin{lstlisting}[language=Java]
public static void main(String[] args) {
    Hello h = new Hello();
    System.out.println(h.getGreeting());
}
\end{lstlisting}
makes the {\tt Hello} class executable.
\begin{itemize}
\item The signature must match {\tt public static void main(String[] args)}
\item It means that we can run the class with the {\tt java} command
\end{itemize}


\end{frame}
%------------------------------------------------------------------------

%------------------------------------------------------------------------
\begin{frame}[fragile]{Compiling a Java Program}


Given the definition of the {\tt Hello} class, we can compile it like this:
\begin{lstlisting}[language=Java]
$ javac Hello.java
\end{lstlisting}

\begin{itemize}
\item \$ is the command prompt.
\item The argument to the {\tt javac} command is the name of a Java source file ending in {\tt .java}
\item This example must be executed in the same directory {\tt Hello.java} is located
\item Executing the command above produces a {\tt Hello.class} file (provided there are no compile errors)
\end{itemize}

\end{frame}
%------------------------------------------------------------------------


%------------------------------------------------------------------------
\begin{frame}[fragile]{Running a Java Program}


Since {\tt Hello} has a {\tt main} method, we can run it:
\begin{lstlisting}[language=Java]
$ java Hello
\end{lstlisting}
\begin{itemize}
\item The argument to the {\tt java} command is the name of a compiled class
\item Java finds this class on the classpath (which includes the current directory by default) and executes its {\tt main} method
\item This example must be executed in the same directory {\tt Hello.class} is located
\end{itemize}



\end{frame}
%------------------------------------------------------------------------

%------------------------------------------------------------------------
\begin{frame}[fragile]{Maintaining Class Invariants}
\vspace{-.05in}
This {\tt Card} class won't allow a client to set an invalid rank:
\vspace{-.05in}
\begin{lstlisting}[language=Java]
public class Card {
    private final String[] VALID_RANKS =
        {"2", "3", "4", "5", "6", "7", "8", "9",
         "10", "jack", "queen", "king", "ace"};
    private String rank;

    public void setRank(String rank) {
        if (!isValidRank(rank)) {
            throw new IllegalArgumentException("Invalid rank.");
        }
        this.rank = rank;
    }
    private boolean isValidRank(String someRank) {
        return Arrays.asList(VALID_RANKS).contains(someRank);
    }
}
\end{lstlisting}
\vspace{-.05in}
\begin{itemize}
\item Types restrict the values you may set for a variable to a particular domain.
\item With encapsulation you can further restrict the domain of allowable values for a variable.
\end{itemize}

\end{frame}
%------------------------------------------------------------------------


%------------------------------------------------------------------------
\begin{frame}[fragile]{Static versus Non-Static}

Consider \link{\code/classes/Doberman.java}{Doberman.java}:
\begin{lstlisting}[language=Java]
public class Doberman {

    private static int dobieCount = 0;

    private String name;

    public Doberman(String name) {
        this.name = name;
        dobieCount++;
    }
    public String reportDobieCount() {
        return name + " says there are " + dobieCount + " dobies.";
    }
}
\end{lstlisting}

\begin{itemize}
\item {\tt dobieCount} is shared between all instances of the {\tt Doberman} class.
\item Each instance of {\tt Doberman} has its own distinct copy of {\tt name}.
\end{itemize}


\end{frame}
%------------------------------------------------------------------------

%------------------------------------------------------------------------
\begin{frame}[fragile]{Reference Counting with a Static Variable}

Given \link{\code/classes/DaringDobermans.java}{DaringDobermans.java}:
\begin{lstlisting}[language=Java]
public class DaringDobermans {
    public static void main(String[] args) {
        Doberman fido = new Doberman("Fido");
        Doberman prince = new Doberman("Prince");
        Doberman chloe = new Doberman("Chloe");
        System.out.println(fido.reportDobieCount());
        System.out.println(prince.reportDobieCount());
        System.out.println(chloe.reportDobieCount());
    }
}
\end{lstlisting}
and our definition of {\tt Doberman}, what will {\tt java DaringDobermans} print?
\begin{lstlisting}[language=Java]
$ java DaringDobermans
Fido says there are 3 dobies.
Prince says there are 3 dobies.
Chloe says there are 3 dobies.
\end{lstlisting}

\end{frame}
%------------------------------------------------------------------------

%------------------------------------------------------------------------
\begin{frame}[fragile]{Not Reference Counting with a Non-Static Variable}

Now remove {\tt static} from the definition of {\tt dobieCount}:
\begin{lstlisting}[language=Java]
    private int dobieCount = 0;
\end{lstlisting}
Now when we run {\tt DaringDobermans} we get
\begin{lstlisting}[language=Java]
$ java DaringDobermans
Fido says there are 1 dobies.
Prince says there are 1 dobies.
Chloe says there are 1 dobies.
\end{lstlisting}

The difference is that now each {\tt Doberman} instance has its own copy of {\tt dobieCount}, not a class-wide, or {\tt static} {\tt dobieCount}


\end{frame}
%------------------------------------------------------------------------

%------------------------------------------------------------------------
\begin{frame}[fragile]{Static Variables}

\vspace{-.05in}
\begin{lstlisting}[language=Java]
public class Doberman {
    private static int dobieCount = 0;
    private String name;

    public Doberman(String name) {
        this.name = name;
        dobieCount++;
    }
    public String reportDobieCount() {
        return name+" says there are "+dobieCount+" dobies.";
    }
}
\end{lstlisting}
\vspace{-.05in}
What does this code print?
\vspace{-.05in}
\begin{lstlisting}[language=Java]
        Doberman fido = new Doberman("Fido");
        System.out.println(fido.reportDobieCount());

        Doberman prince = new Doberman("Prince");
        System.out.println(prince.reportDobieCount());

        Doberman chloe = new Doberman("Chloe");
        System.out.println(chloe.reportDobieCount());
\end{lstlisting}


\end{frame}
%------------------------------------------------------------------------



%------------------------------------------------------------------------
\begin{frame}[fragile]{Review Questions: Variables and Values}

\begin{itemize}
\item Is {\tt int n = 2.2;} legal?
\item What's the value of the expression {\tt 17 \% 4}?
\item What's the value of {\tt n} after {\tt int n = (int) 2.2;}?
\item After the line above and {\tt n++;}, what's the value of {\tt n}?
\item After the line above and {\tt n += 2;}, what's the value of {\tt n}?
\item After the line above and {\tt String s = "Answer: " + n;}, what's the value of {\tt s}?
\end{itemize}
Given
\begin{lstlisting}[language=Java]
boolean a = true;
boolean b = false;
\end{lstlisting}
\vspace{-.1in}
\begin{itemize}
\item What's the value of {\tt x} after {\tt boolean x = a || b;}?
\item What's the value of {\tt y} after {\tt boolean y = a \&\& b;}?
\end{itemize}


\end{frame}
%------------------------------------------------------------------------

%------------------------------------------------------------------------
\begin{frame}[fragile]{Review Questions: {\tt if}-Statements}


Will this code compile?

\begin{lstlisting}[language=Java]
String condition = "true";
if (condition) {
    System.out.println("The true path.");
} else {
    System.out.println("The false path.");
}
\end{lstlisting}

What will this code print?
\begin{lstlisting}[language=Java]
boolean a = true;
boolean b = false;
if (a && b ) {
    System.out.println("The true path.");
} else {
    System.out.println("The false path.");
}
\end{lstlisting}

\end{frame}
%------------------------------------------------------------------------

%------------------------------------------------------------------------
\begin{frame}[fragile]{Short-Circuit Evaluation}


What will this code print?
\vspace{-.05in}
\begin{lstlisting}[language=Java]
public class ShortCircuit {

    private static int counter = 0;

    private static boolean incrementCounter() {
        counter++;
        return true;
    }
    public static void main(String args[]) {
        boolean a = true;
        boolean b = false;
        if (a || incrementCounter()) {
            System.out.println("Evaluated (a || incrementCounter()).");
        }
        System.out.println("Counter = " + counter);
        if (a && incrementCounter()) {
            System.out.println("Evaluated (a && incrementCounter()).");
        }
        System.out.println("Counter = " + counter);
    }
}
\end{lstlisting}


\end{frame}
%------------------------------------------------------------------------

%------------------------------------------------------------------------
\begin{frame}[fragile]{Review Qustions: Loops}


How would you write this {\tt while} loop as a {\tt for} loop?
\begin{lstlisting}[language=Java]
int n = 0;
while (n < 5) {
    System.out.println("Hip, hip, hooray!");
    n++;
}
\end{lstlisting}

Answer:
\begin{lstlisting}[language=Java]
for (int n = 0; n < 5; n++) {
    System.out.println("Hip, hip, hooray!");
}
\end{lstlisting}


\end{frame}
%------------------------------------------------------------------------

%------------------------------------------------------------------------
\begin{frame}[fragile]{Equality}
\vspace{-.05in}
\begin{itemize}
\item To compare objects, like {\tt String}s, for value equality, use {\tt equals}.
\item For objects, {\tt ==} means identity equality -- are two references aliases to the same object in memory.
\end{itemize}

What will this code print?
\begin{lstlisting}[language=Java]
public class Foo {
    private String bar;
    public Foo(String bar) {
        this.bar = bar;
    }
    public boolean equals(Object other) {
        return this.bar.equals(((Foo) other).bar);
    }
    public static void main(String[] args) {
        Foo foo1 = new Foo("bar");
        Foo foo2 = new Foo("bar");
        Foo foo3 = foo1;
        System.out.println("foo1.equals(foo2): " + foo1.equals(foo2));
        System.out.println("foo1 == foo2: " + (foo1 == foo2));
        System.out.println("foo1 == foo3: " + (foo1 == foo3));
    }
}
\end{lstlisting}

\end{frame}
%------------------------------------------------------------------------

% %------------------------------------------------------------------------
% \begin{frame}[fragile]{}


% \begin{lstlisting}[language=Java]

% \end{lstlisting}

% \begin{itemize}
% \item
% \end{itemize}


% \end{frame}
% %------------------------------------------------------------------------


\end{document}
