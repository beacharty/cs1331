\documentclass{beamer}

\newcommand{\lesson}{Streams}


\newcommand{\course}{Introduction to Object-Oriented Programming}
\subject{\course}
\title[\lesson]{\course}
\subtitle{\lesson}

\author[CS 1331]
{Christopher Simpkins \\\texttt{chris.simpkins@gatech.edu}}
\institute[Georgia Tech]

\date[]{}

\newcommand{\link}[2]{\href{#1}{\textcolor{blue}{\underline{#2}}}}
\newcommand{\code}{http://www.cs1331.org/code}

\usepackage{colortbl}

% If you have a file called "university-logo-filename.xxx", where xxx
% is a graphic format that can be processed by latex or pdflatex,
% resp., then you can add a logo as follows:

% \pgfdeclareimage[width=0.6in]{coc-logo}{cc_2012_logo}
% \logo{\pgfuseimage{coc-logo}}

\mode<presentation>
{
  \usetheme{Berlin}
  \useoutertheme{infolines}

  % or ...

 \setbeamercovered{transparent}
  % or whatever (possibly just delete it)
}

\usepackage{tikz}
% Optional PGF libraries
\usepackage{pgflibraryarrows}
\usepackage{pgflibrarysnakes}
\usepackage{pgfplots}
\usepackage{fancybox}
\usepackage{listings}
\usepackage{hyperref}
\hypersetup{colorlinks=true,urlcolor=blue}
\usepackage[english]{babel}
% or whatever

\usepackage[latin1]{inputenc}
% or whatever

\usepackage{times}
\usepackage[T1]{fontenc}
% Or whatever. Note that the encoding and the font should match. If T1
% does not look nice, try deleting the line with the fontenc.


\usepackage{listings}

% "define" Scala
\lstdefinelanguage{scala}{
  morekeywords={abstract,case,catch,class,def,%
    do,else,extends,false,final,finally,%
    for,if,implicit,import,match,mixin,%
    new,null,object,override,package,%
    private,protected,requires,return,sealed,%
    super,this,throw,trait,true,try,%
    type,val,var,while,with,yield},
  otherkeywords={=>,<-,<\%,<:,>:,\#,@},
  sensitive=true,
  morecomment=[l]{//},
  morecomment=[n]{/*}{*/},
  morestring=[b]",
  morestring=[b]',
  morestring=[b]""",
}

\usepackage{color}
\definecolor{dkgreen}{rgb}{0,0.6,0}
\definecolor{gray}{rgb}{0.5,0.5,0.5}
\definecolor{mauve}{rgb}{0.58,0,0.82}

% Default settings for code listings
\lstset{frame=tb,
  language=scala,
  aboveskip=2mm,
  belowskip=2mm,
  showstringspaces=false,
  columns=flexible,
  basicstyle={\scriptsize\ttfamily},
  numbers=none,
  numberstyle=\tiny\color{gray},
  keywordstyle=\color{blue},
  commentstyle=\color{dkgreen},
  stringstyle=\color{mauve},
  frame=single,
  breaklines=true,
  breakatwhitespace=true,
  keepspaces=true
  %tabsize=3
}


% If you wish to uncover everything in a step-wise fashion, uncomment
% the following command:

% \beamerdefaultoverlayspecification{<+->}


\begin{document}

\begin{frame}
  \titlepage
\end{frame}

%------------------------------------------------------------------------
\begin{frame}[fragile]{Streams and Pipelines}

A stream is a sequence of elements.
\begin{itemize}
\item Unlike a collection, it is not a data structure that stores elements.
\item Unlike an iterator, streams do not allow modification of the underlying source
\end{itemize}

A stream carries values from a source through a pipeline.\\

\vspace{.1in}

A pipeline contains the following components:

\begin{itemize}
\item A source: This could be a collection, an array, a generator function, or an I/O channel.
\item Zero or more intermediate operations. An intermediate operation, such as filter, produces a new stream
\item A terminal operation. A terminal operation, such as forEach, produces a non-stream result, such as a primitive value (like a double value), a collection, or in the case of forEach, no value at all.
\end{itemize}

\end{frame}
%------------------------------------------------------------------------

%------------------------------------------------------------------------
\begin{frame}[fragile]{Stream Example: How Many Mustaches?}

Consider this simple example from \link{\code/collections/SuperTroopers.java}{SuperTroopers.java}:
\begin{lstlisting}[language=Java]
long mustaches = troopers.stream().filter(Trooper::hasMustache).count();
System.out.println("Mustaches: " + mustaches);
\end{lstlisting}

\begin{itemize}
\item {\tt troopers.stream()} is the {\it source}
\item {\tt .filter(Trooper::hasMustache)} is an {\it intermediate operation}
\item {\tt .count()} is the {\it terminal operation}
\end{itemize}
The terminal operation yields a new value which results from applying all the intermediate operations and finally the terminal operation to the source.

\end{frame}
%------------------------------------------------------------------------

%------------------------------------------------------------------------
\begin{frame}[fragile]{A Bigger Stream Example: WordCount Pipeline}

Consider this example from {\tt WordCount}s:
\begin{lstlisting}[language=Java]
Set<String> stopWords = new HashSet<>(Arrays.asList(
    "a", "an", "and", "are", "as", "be", "by", "is", "in", "of",
    "for", "from", "not", "to", "the", "that", "this", "with", "which"
));
wc.wordCounts.entrySet().stream()
    .filter(entry -> !stopWords.contains(entry.getKey().toLowerCase()))
    .sorted((e1, e2) -> e1.getValue() - e2.getValue())
    .forEach(entry ->
        System.out.printf("%s occurs %d times%n", entry.getKey(),
                                                  entry.getValue()));
\end{lstlisting}

This code does the same tasks we did before with classes and for loops.

\end{frame}
%------------------------------------------------------------------------

%------------------------------------------------------------------------
\begin{frame}[fragile]{WordCount Pipeline - Stop Words}

\begin{lstlisting}[language=Java]
Set<String> stopWords = new HashSet<>(Arrays.asList(
    "a", "an", "and", "are", "as", "be", "by", "is", "in", "of",
    "for", "from", "not", "to", "the", "that", "this", "with", "which"
));

\end{lstlisting}


\begin{itemize}
\item Every document has information-carrying words and grammatical words that carry no information, like prepositions, verbs like to be or have, pronouns
\item In document processing we call these non-information-carrying words {\it stop words}
\end{itemize}

Here we've implemented a naiive and terribly incomplete stop words list.\\

\vspace{.1in}

BTW, why a {\tt HashSet}?

\end{frame}
%------------------------------------------------------------------------

%------------------------------------------------------------------------
\begin{frame}[fragile]{WordCount Pipeline - filter}

Consider this example from {\tt WordCount}s:
\begin{lstlisting}[language=Java]
Set<String> stopWords = new HashSet<>(Arrays.asList(
    "a", "an", "and", "are", "as", "be", "by", "is", "in", "of",
    "for", "from", "not", "to", "the", "that", "this", "with", "which"
));
wc.wordCounts.entrySet().stream()
    .filter(entry -> !stopWords.contains(entry.getKey().toLowerCase()))
    .sorted((e1, e2) -> e1.getValue() - e2.getValue())
    .forEach(entry ->
        System.out.printf("%s occurs %d times%n", entry.getKey(),
                                                  entry.getValue()));
\end{lstlisting}

The filter operation takes a predicate function.
\begin{itemize}
\item A predicate function returns a {\tt boolean}
\item If predicate function returns {\tt true}, element is retained in the stream
\end{itemize}
Notice that we're also normalizing words to lower case.

\end{frame}
%------------------------------------------------------------------------

%------------------------------------------------------------------------
\begin{frame}[fragile]{WordCount Pipeline - sorted}

Consider this example from {\tt WordCount}s:
\begin{lstlisting}[language=Java]
Set<String> stopWords = new HashSet<>(Arrays.asList(
    "a", "an", "and", "are", "as", "be", "by", "is", "in", "of",
    "for", "from", "not", "to", "the", "that", "this", "with", "which"
));
wc.wordCounts.entrySet().stream()
    .filter(entry -> !stopWords.contains(entry.getKey().toLowerCase()))
    .sorted((e1, e2) -> e1.getValue() - e2.getValue())
    .forEach(entry ->
        System.out.printf("%s occurs %d times%n", entry.getKey(),
                                                  entry.getValue()));
\end{lstlisting}

The sorted operation takes a {\tt Comparator} that defines the ordering over the stream's elements.

\end{frame}
%------------------------------------------------------------------------

%------------------------------------------------------------------------
\begin{frame}[fragile]{WordCount Pipeline - forEach}

Consider this example from {\tt WordCount}s:
\begin{lstlisting}[language=Java]
Set<String> stopWords = new HashSet<>(Arrays.asList(
    "a", "an", "and", "are", "as", "be", "by", "is", "in", "of",
    "for", "from", "not", "to", "the", "that", "this", "with", "which"
));
wc.wordCounts.entrySet().stream()
    .filter(entry -> !stopWords.contains(entry.getKey().toLowerCase()))
    .sorted((e1, e2) -> e1.getValue() - e2.getValue())
    .forEach(entry ->
        System.out.printf("%s occurs %d times%n", entry.getKey(),
                                                  entry.getValue()));
\end{lstlisting}

{\tt forEach} is the terminal operation.
\begin{itemize}
\item Called for its effect - no return value
\end{itemize}

The underlying source is not modified.

\end{frame}
%------------------------------------------------------------------------


%------------------------------------------------------------------------
\begin{frame}[fragile]{Homework: Make the WordCount Pipeline Clearer}

Notice that we use anonymous lambda expressions in our WordCOunt pipeline:
\begin{lstlisting}[language=Java]
wc.wordCounts.entrySet().stream()
    .filter(entry -> !stopWords.contains(entry.getKey().toLowerCase()))
    .sorted((e1, e2) -> e1.getValue() - e2.getValue())
    .forEach(entry ->
        System.out.printf("%s occurs %d times%n", entry.getKey(),
                                                  entry.getValue()));
\end{lstlisting}

\begin{itemize}
\item Functional-style code can easily become hard to read.
\item You can improve readability by introducing intermediate helper variables with informative names.
\end{itemize}
Rewrite the WordCount pipeline with intermediate helper variables so that the pipeline is easy to understand.  You'll need to look up these aggregate operations in the Java API to get the types for these variables.

\end{frame}
%------------------------------------------------------------------------


% %------------------------------------------------------------------------
% \begin{frame}[fragile]{}


% \begin{lstlisting}[language=Java]

% \end{lstlisting}

% \begin{itemize}
% \item
% \end{itemize}


% \end{frame}
% %------------------------------------------------------------------------


\end{document}
