\documentclass{beamer}

\newcommand{\lesson}{Arrays}


\newcommand{\course}{Introduction to Object-Oriented Programming}
\subject{\course}
\title[\lesson]{\course}
\subtitle{\lesson}

\author[CS 1331]
{Christopher Simpkins \\\texttt{chris.simpkins@gatech.edu}}
\institute[Georgia Tech]

\date[]{}

\newcommand{\link}[2]{\href{#1}{\textcolor{blue}{\underline{#2}}}}
\newcommand{\code}{http://www.cs1331.org/code}

\usepackage{colortbl}

% If you have a file called "university-logo-filename.xxx", where xxx
% is a graphic format that can be processed by latex or pdflatex,
% resp., then you can add a logo as follows:

% \pgfdeclareimage[width=0.6in]{coc-logo}{cc_2012_logo}
% \logo{\pgfuseimage{coc-logo}}

\mode<presentation>
{
  \usetheme{Berlin}
  \useoutertheme{infolines}

  % or ...

 \setbeamercovered{transparent}
  % or whatever (possibly just delete it)
}

\usepackage{tikz}
% Optional PGF libraries
\usepackage{pgflibraryarrows}
\usepackage{pgflibrarysnakes}
\usepackage{pgfplots}
\usepackage{fancybox}
\usepackage{listings}
\usepackage{hyperref}
\hypersetup{colorlinks=true,urlcolor=blue}
\usepackage[english]{babel}
% or whatever

\usepackage[latin1]{inputenc}
% or whatever

\usepackage{times}
\usepackage[T1]{fontenc}
% Or whatever. Note that the encoding and the font should match. If T1
% does not look nice, try deleting the line with the fontenc.


\usepackage{listings}

% "define" Scala
\lstdefinelanguage{scala}{
  morekeywords={abstract,case,catch,class,def,%
    do,else,extends,false,final,finally,%
    for,if,implicit,import,match,mixin,%
    new,null,object,override,package,%
    private,protected,requires,return,sealed,%
    super,this,throw,trait,true,try,%
    type,val,var,while,with,yield},
  otherkeywords={=>,<-,<\%,<:,>:,\#,@},
  sensitive=true,
  morecomment=[l]{//},
  morecomment=[n]{/*}{*/},
  morestring=[b]",
  morestring=[b]',
  morestring=[b]""",
}

\usepackage{color}
\definecolor{dkgreen}{rgb}{0,0.6,0}
\definecolor{gray}{rgb}{0.5,0.5,0.5}
\definecolor{mauve}{rgb}{0.58,0,0.82}

% Default settings for code listings
\lstset{frame=tb,
  language=scala,
  aboveskip=2mm,
  belowskip=2mm,
  showstringspaces=false,
  columns=flexible,
  basicstyle={\scriptsize\ttfamily},
  numbers=none,
  numberstyle=\tiny\color{gray},
  keywordstyle=\color{blue},
  commentstyle=\color{dkgreen},
  stringstyle=\color{mauve},
  frame=single,
  breaklines=true,
  breakatwhitespace=true,
  keepspaces=true
  %tabsize=3
}


% \beamerdefaultoverlayspecification{<+->}

\begin{document}

\begin{frame}
  \titlepage
\end{frame}


%------------------------------------------------------------------------
\begin{frame}[fragile]{Modeling Aggregates}


As you've seen, you can get pretty far with "scalar" data.  But many phenomena we wish to model computationally are aggregates, or collections, for example:
\begin{itemize}
\item scores on assignments in a class,
\item word counts in a document, or
\item pixel colors in a bitmap image.
\end{itemize}

Today we'll learn Java's most basic facility for modeling aggregates: arrays.

\end{frame}
%------------------------------------------------------------------------


%------------------------------------------------------------------------
\begin{frame}[fragile]{Arrays}


Java Arrays (\href{http://docs.oracle.com/javase/specs/jls/se8/html/jls-10.html}{JLS \S 10}):
\begin{itemize}
\item are objects,
\item are dynamically allocated (e.g., with operator {\tt new}), and
\item have a fixed number of elements of the same type.
\end{itemize}

\end{frame}
%------------------------------------------------------------------------

%------------------------------------------------------------------------
\begin{frame}[fragile]{Creating Arrays}


Consider the following {\it array creation expression} (\href{http://docs.oracle.com/javase/specs/jls/se8/html/jls-10.html#jls-10.3}{JLS \S 10.3)}:
\begin{lstlisting}[language=Java]
double[] scores = new double[5];
\end{lstlisting}
This declaration:
\begin{itemize}
\item allocates a 5-element array,
\item the {\tt 5} in the example above can be any expression that is unary promotable to an {\tt int} (\href{http://docs.oracle.com/javase/specs/jls/se8/html/jls-5.html#jls-5.6.1}{JLS \S 5.6.1})
\item stores the address of this new array in {\tt scores}, and
\item initializes each value to it's default value ({\tt 0} for numeric types, {\tt false} for {\tt boolean} types, and {\tt null} for references, \href{http://docs.oracle.com/javase/specs/jls/se8/html/jls-4.html#jls-4.12.5}{JLS \S 4.12.5}).
\end{itemize}


\end{frame}
%------------------------------------------------------------------------



%------------------------------------------------------------------------
\begin{frame}[fragile]{Array Declarations}


The preceding array definition
\begin{lstlisting}[language=Java]
double[] scores = new double[5];
\end{lstlisting}
could be split into a declaration and initialization:
\begin{lstlisting}[language=Java]
double[] scores;
scores = new double[5];
\end{lstlisting}
Also, you can put the {\tt []} on the type or the variable name when delaring an array.  These two declarations are equivalent:
\begin{lstlisting}[language=Java]
double[] scores;
double scores[];
\end{lstlisting}
Generally, it's better style to put the {\tt []} on the type.


\end{frame}
%------------------------------------------------------------------------

%------------------------------------------------------------------------
\begin{frame}[fragile]{Mixed Declarations}


Note that you can mix aray declarations with declarations of variables having the same element type.  The declaration line:
\begin{lstlisting}[language=Java]
double scores[], average;
\end{lstlisting}
creates
\begin{itemize}
\item an array of {\tt double} reference named {\tt scores}, and
\item a {\tt double} variable named {\tt average}
\end{itemize}

What's the size of the {\tt scores} array declared above?

\end{frame}
%------------------------------------------------------------------------

%------------------------------------------------------------------------
\begin{frame}[fragile]{Array Objects}


After the definition:
\begin{lstlisting}[language=Java]
double[] scores = new double[5];
\end{lstlisting}
{\tt scores} points to an array object in memory that can be visualized as:\\
\vspace{.1in}
\begin{tabular}{p{.5in}p{.5in}p{.5in}p{.5in}p{.5in}}
0 & 1 & 2 & 3 & 4 \\
\end{tabular}
\begin{tabular}{|p{.5in}|p{.5in}|p{.5in}|p{.5in}|p{.5in}|}\hline
0.0 & 0.0 & 0.0 & 0.0 & 0.0 \\
\hline
\end{tabular}\\
\vspace{.1in}
The {\it indexes} of {\tt scores} range from 0 to 4.  The size of arrays are stored in a {\tt public final} instance variable named {\tt length}

\begin{lstlisting}[language=Java]
scores.length == 5;
\end{lstlisting}
What is the type and value of the expression above?

\end{frame}
%------------------------------------------------------------------------

%------------------------------------------------------------------------
\begin{frame}[fragile]{Accessing Array Elements}


Array elements are accessed with an {\tt int}-promotable expression enclosed in square brackets ({\tt []})
\begin{lstlisting}[language=Java]
double[] scores = new double[5];
scores[0] = 89;
scores[1] = 100;
scores[2] = 95.6;
scores[3] = 84.5;
scores[4] = 91;
scores[scores.length - 1] = 99.2;
\end{lstlisting}


\vspace{.1in}
Will this line compile?  If so, what will happen at runtime?
\begin{lstlisting}[language=Java]
scores[scores.length] = 100;
\end{lstlisting}


\end{frame}
%------------------------------------------------------------------------

%------------------------------------------------------------------------
\begin{frame}[fragile]{Initializing Arrays}


You can provide initial values for (small) arrays
\begin{lstlisting}[language=Java]
String[] validSuits = {"diamonds", "clubs", "hearts","spades"};
\end{lstlisting}

\begin{itemize}
\item What is {\tt validSuits.length}?
\item What is {\tt validSuits[1]}?
\end{itemize}

You can also use a loop to initialize the values of an array:
\begin{lstlisting}[language=Java]
int[] squares = new int[5];
for (int i = 0; i < squares.length; ++i) {
    squares[i] = i*i;
}
\end{lstlisting}

What is {\tt squares[4]}?

\end{frame}
%------------------------------------------------------------------------

%------------------------------------------------------------------------
\begin{frame}[fragile]{Traversing Arrays}


Arrays and {\tt for} statements go hand-in-hand:
\begin{lstlisting}[language=Java]
double[] scores = new double[5];
for (int i = 0; i < 5; ++i) {
    System.out.printf("scores[%d] = %.2f%n", i, scores[i]);
}
\end{lstlisting}

You can also use the ``enhanced'' for loop:

\begin{lstlisting}[language=Java]
for (double score: scores) {
    System.out.println(score);
}
\end{lstlisting}

Read the enhanced for loop as ``for each element of the array ...''.\\
\vspace{.1in}
Why use for-each instead of traditional for? ...

\end{frame}
%------------------------------------------------------------------------

%------------------------------------------------------------------------
\begin{frame}[fragile]{Traditional {\tt for} Versus {\tt for}-each}


In cases where you don't need the index, use the enhanced for loop.  Consider:
\vspace{-.05in}
\begin{lstlisting}[language=Java]
double sum = 0.0;
for (int i = 0; i < scores.length; ++i) {
    sum += scores[i];
}
\end{lstlisting}
\vspace{-.05in}
In the code above, {\tt scores.length} is used only for bounding the array traversal, and the index {\tt i} is only used for sequential array access.  Those are two things we can mess up.  The enhanced for loop is cleaner:

\begin{lstlisting}[language=Java]
double sum = 0.0;
for (double score: scores) {
    sum += score;
}
\end{lstlisting}

Also note how our naming conventions help to make the code clear.  You can read the loop above as ``for each score in scores''.

\end{frame}
%------------------------------------------------------------------------

%------------------------------------------------------------------------
\begin{frame}[fragile]{Array Initialization and Access Gotchas}


Because arrays are allocated dynamically, this will compile:
\begin{lstlisting}[language=Java]
double[] scores = new double[-5];
\end{lstlisting}
but will produce an error at run-time:
\begin{lstlisting}[language=Java]
Exception in thread "main" java.lang.NegativeArraySizeException
        at ArrayBasics.main(ArrayBasics.java:4)
\end{lstlisting}

Also, array access expressions are evaluated and checked at run-time.  So, in the same way that accessing an array with an index $\ge$ the size of the array produces a run-time error, negative indexes like:

\begin{lstlisting}[language=Java]
scores[-1] = 100;
\end{lstlisting}
produce:
\begin{lstlisting}[language=Java]
Exception in thread "main" java.lang.ArrayIndexOutOfBoundsException: -1
        at ArrayBasics.main(ArrayBasics.java:23)
\end{lstlisting}
\end{frame}
%------------------------------------------------------------------------

%------------------------------------------------------------------------
\begin{frame}[fragile]{Arrays as Method Parameters - {\tt main}}


We've already seen an array parameter:
\begin{lstlisting}[language=Java]
public static void main(String[] args)
\end{lstlisting}

We can use this array just like we use any other array.
\begin{lstlisting}[language=Java]
public class Shout {

    public static void main(String[] args) {
        for (String arg: args) {
            System.out.print(arg.toUpperCase() + " ");
        }
        System.out.println();
    }
}
\end{lstlisting}
See also \link{\code/arrays/CourseAverage.java}{CourseAverage.java}
\end{frame}
%------------------------------------------------------------------------

%------------------------------------------------------------------------
\begin{frame}[fragile]{Variable Arity Parameters}


\begin{itemize}
\item The {\it arity} of a method is its number of formal parameters.
\item So far, all the methods we've written have fixed arity.
\item The last parameter to a method may be a {\it variable arity parameter}, a.k.a. {\it var args} parameter (\href{http://docs.oracle.com/javase/specs/jls/se8/html/jls-8.html#jls-8.4.1}{JLS \S 8.4.1}), whose syntax is simply to add an ellipse ({\tt ...}) after the type name.
\item The var args parameter is accessed as an array inside the method.
\end{itemize}
For example:
\begin{lstlisting}[language=Java]
public static int max(int ... numbers) {
    int max = numbers[0];
    for (int i = 1; i < numbers.length; ++i) {
        if (numbers[i] > max) max = number;
    }
    return max;
}
\end{lstlisting}

\end{frame}
%------------------------------------------------------------------------

%------------------------------------------------------------------------
\begin{frame}[fragile]{Multidimensional Arrays}


You can create arrays of any number of dimensions simply by adding additional square brackets for dimensions and sizes.  For example:

\begin{lstlisting}[language=Java]
char[][] grid;
\end{lstlisting}
The declaration statement above:
\begin{itemize}
\item Declares a 2-dimensional array of  {\tt char}.
\item As with one-dimensinal arrays, {\tt char} is the base type.
\item Each element of {\tt grid}, which is indexed by two {\tt int} expressions, is a {\tt char} variable.
\end{itemize}


\end{frame}
%------------------------------------------------------------------------

%------------------------------------------------------------------------
\begin{frame}[fragile]{Initializing Multidimensional Arrays}


Initialization of 2-dimensional arrays can be done with {\tt new}:
\begin{lstlisting}[language=Java]
grid = new char[10][10];
\end{lstlisting}

or with literal initialization syntax:
\begin{lstlisting}[language=Java]
char[][] grid = {{' ', ' ', ' ', ' ', ' ', ' ', ' ', ' ', ' ', ' '},
                 {' ', ' ', ' ', ' ', ' ', ' ', ' ', ' ', ' ', ' '},
                 {' ', '*', '*', ' ', ' ', ' ', ' ', '*', '*', ' '},
                 {' ', '*', '*', ' ', ' ', ' ', ' ', '*', '*', ' '},
                 {' ', ' ', ' ', ' ', '*', '*', ' ', ' ', ' ', ' '},
                 {' ', ' ', ' ', ' ', '*', '*', ' ', ' ', ' ', ' '},
                 {' ', '*', ' ', ' ', ' ', ' ', ' ', ' ', '*', ' '},
                 {' ', ' ', '*', ' ', ' ', ' ', ' ', '*', ' ', ' '},
                 {' ', ' ', ' ', '*', '*', '*', '*', ' ', ' ', ' '},
                 {' ', ' ', ' ', ' ', ' ', ' ', ' ', ' ', ' ', ' '}};
\end{lstlisting}

Notice that a 2-dimensional array is an array of 1-dimensional arrays (and a 3-dimensional array is an array of 2-dimensional arrays, and so on).

\end{frame}
%------------------------------------------------------------------------

%------------------------------------------------------------------------
\begin{frame}[fragile]{Visualizing Multidimensional Arrays}


Our 2-dimensional {\tt grid} array can be visualized as a 2-d grid of cells.\\
\vspace{.1in}
\begin{tabular}{p{.4in}p{.2in}p{.2in}p{.2in}p{.2in}p{.2in}p{.2in}p{.2in}p{.2in}p{.2in}p{.2in}}
         & [0] & [1] & [2] & [3] & [4] & [5] & [6] & [7] & [8] & [9]
\end{tabular}
\begin{tabular}{p{.4in}|p{.2in}|p{.2in}|p{.2in}|p{.2in}|p{.2in}|p{.2in}|p{.2in}|p{.2in}|p{.2in}|p{.2in}|}\cline{2-11}
grid[0] & ' ' & ' ' & ' ' & ' ' & ' ' & ' ' & ' ' & ' ' & ' ' & ' ' \\
\cline{2-11}
grid[1] & ' ' & ' ' & ' ' & ' ' & ' ' & ' ' & ' ' & ' ' & ' ' & ' ' \\
\cline{2-11}
grid[2] & ' ' & '*' & '*' & ' ' & ' ' & ' ' & ' ' & '*' & '*' & ' ' \\
\cline{2-11}
grid[3] & ' ' & '*' & \cellcolor{yellow}'*' & ' ' & ' ' & ' ' & ' ' & '*' & '*' & ' ' \\
\cline{2-11}
grid[4] & ' ' & ' ' & ' ' & ' ' & '*' & '*' & ' ' & ' ' & ' ' & ' ' \\
\cline{2-11}
grid[5] & ' ' & ' ' & ' ' & ' ' & '*' & '*' & ' ' & ' ' & ' ' & ' ' \\
\cline{2-11}
grid[6] & ' ' & '*' & ' ' & ' ' & ' ' & ' ' & ' ' & ' ' & '*' & ' ' \\
\cline{2-11}
grid[7] & ' ' & ' ' & '*' & ' ' & ' ' & ' ' & ' ' & '*' & ' ' & ' ' \\
\cline{2-11}
grid[8] & ' ' & ' ' & ' ' & '*' & '*' & '*' & '*' & ' ' & ' ' & ' ' \\
\cline{2-11}
grid[9] & ' ' & ' ' & ' ' & ' ' & ' ' & ' ' & ' ' & ' ' & ' ' & ' ' \\
\cline{2-11}
\end{tabular}\\
\vspace{.1in}
And an individual cell can be accessed by supplying two indices:

\begin{lstlisting}[language=Java]
grid[3][2] == '*'; // true
\end{lstlisting}

\end{frame}
%------------------------------------------------------------------------

%------------------------------------------------------------------------
\begin{frame}[fragile]{Traversing Multidimensional Arrays}


Traverse 2-dimensional array by nesting loops.  The key to getting it right is to use the right {\tt length}s.
\vspace{-.05in}
\begin{lstlisting}[language=Java]
for (int row = 0; row < grid.length; ++row) {
    for (int col = 0; col < grid[row].length; ++col) {
        System.out.print(grid[row][col]);
    }
    System.out.println();
}
\end{lstlisting}
Note that the for loops above traverse the grid in row-major order.  We can traverse the grid in column-major order by reversing the nesting of the for loops:
\vspace{-.05in}
\begin{lstlisting}[language=Java]
for (int col = 0; col < grid[0].length; ++col) {
    for (int row = 0; row < grid.length; ++row) {
        System.out.print(grid[row][col]);
    }
    System.out.println();
}
\end{lstlisting}
\vspace{-.05in}
See \link{\code/arrays/Smiley.java}{Smiley.java}

\end{frame}
%------------------------------------------------------------------------

%------------------------------------------------------------------------
\begin{frame}[fragile]{Closing Thoughts}

\begin{itemize}
\item Arrays are our first ``collection classes'' (but are not Java {\tt Collection} classes).
\item Arrays are objects, so array objects are created with operator {\tt new} and array variables can have the value {\tt null}.
\item Arrays have sugar to add convenience and make them syntactically similar to C's arrays.
\end{itemize}


\end{frame}
%------------------------------------------------------------------------

% %------------------------------------------------------------------------
% \begin{frame}[fragile]{}


% \begin{lstlisting}[language=Java]

% \end{lstlisting}

% \begin{itemize}
% \item
% \end{itemize}


% \end{frame}
% %------------------------------------------------------------------------


\end{document}
